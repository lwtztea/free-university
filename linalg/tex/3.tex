\documentclass[a4paper, 12pt]{article}

\usepackage[T2A]{fontenc}
\usepackage[utf8]{inputenc}
\usepackage[english, russian]{babel}
\usepackage[left=19mm,right=19mm,top=2cm,bottom=2cm,bindingoffset=0cm]{geometry}
\usepackage{amsmath, amsfonts, amssymb, amsthm, mathtools}
\usepackage{enumitem}
\usepackage{setspace}
\usepackage{amsmath}

\begin{document}
\begin{spacing}{1.5}
\setlength{\parindent}{0ex}

\section*{Семинар 3}


\subsection*{Опять немного теории множеств}

\textbf{Определение 3.1.} Пусть $A$ и $B$ -- некоторые множества. Отображение (не обязательно линейное) $f: A \rightarrow B$ называется \textit{инъективным}, или \textit{инъекцией}, если оно переводит разные элементы в разные, т.е. $\forall a_1, a_2 \in A \ a_1 \neq a_2 \Rightarrow f(a_1) \neq f(a_2)$.

\textbf{Определение 3.2.} Отображение $f: A \rightarrow B$ называется \textit{сюръективным}, или \textit{сюръекцией}, если множество его значений есть всё $B$, т.е. $\forall b \in B \ \exists a \in A: f(a) = b$.

\textbf{Определение 3.3.} Отображение $f: A \rightarrow B$, которое одновременно является инъекцией и сюръекцией, называется \textit{биекцией}, или  \textit{взаимо однозначным соответствием}.


\subsection*{Немного теории групп}

\textbf{Определение 3.4.} \textit{Группой} называется множество $G$ с определённой на нём бинарной операцией $\cdot : G \times G \rightarrow G$ такой, что:
\begin{enumerate}[noitemsep]
    \item $\forall a, b, c \in G: a(bc) = (ab)c$.
    \item $\exists e \in G: \forall a \in G: ae = ea = a$ (элемент $e$ называется \textit{нейтральным}).
    \item $\forall a \in G \ \exists a^{-1} \in G: aa^{-1} = a^{-1}a = e$ (элемент $aa^{-1}$ называется \textit{обратным}).
\end{enumerate}

\textbf{Упражнение 3.1.} Осознать, что линейное пространство является группой относительно операции сложения векторов. 

\textbf{Определение 3.5.} Пусть заданы две группы $(G, \cdot)$ и $(H, *)$. \textit{Гомоморфизмом} этих групп называется функция $h: G \rightarrow H$ такая, что $\forall u, v \in G \ h(u \cdot v) = h(u) * h(v)$. Проще говоря, гомоморфизм $h$ сохраняет групповую структуру.

\textbf{Упражнение 3.2.} Доказать, что гомоморфизм переводит нейтральный элемент в нейтральный, а обратные -- в обратные. 

\textbf{Определение 3.6.} \textit{Изоморфизмом} называется биективный гомоморфизм. Группы $G$ и $H$ называются \textit{изоморфными}, если существует соответствующий изоморфизм.

\textbf{Утверждение 3.1.} $\varphi$ -- инъективный гомоморфизм $\Leftrightarrow \text{Ker } \varphi = \{e\}$. (Докажите это!)


\subsection*{Возвращаемся к линейной алгебре}

\textbf{Определение 3.7.} \textit{Изоморфизмом} линейных пространств $U$ и $V$ называется биективное отображение $\varphi: U \rightarrow V$, удовлетворяющее свойствам:
\begin{enumerate}[noitemsep]
    \item $\forall u_1, u_2 \in U: \varphi(u_1 + u_2) = \varphi(u_1) + \varphi(u_2)$.
    \item $\forall \alpha \in \mathbb{R} \ \forall u \in U: \varphi(\alpha u) = \alpha \varphi(u)$.
\end{enumerate}
Пространства $U$ и $V$ называются \textit{изоморфными}, если между ними существует изоморфизм.

\textbf{Утверждение 3.2.} Линейные пространства одной размерности являются изоморфными.

\textbf{Утверждение 3.3.} Изоморфизм есть отошение эквивалентности.


\subsection*{Сумма и пересечение подпространств}

В целом с подпространствами и их пересечениями всё очевидно :)

\textbf{Определение 3.8.} Пусть $V$ -- линейное пространство, $U_1$, $U_2$ -- подпространства $V$. \textit{Суммой} пространств $U_1$ и $U_2$ называется следующее множество:
$$U_1 + U_2 = \{ u_1 + u_2 \ | \ u_1 \in U_1, u_2 \in U_2 \}$$
Аналогично определяется сумма $k$ подпространств  $U_1, ..., U_k \leq V$.

\textbf{Замечание.} Определить сумму $U_1 + ... + U_k$ можно и другим эквивалентным способом: 
$$U_1 + ... + U_k = \langle U_1 \cup ... \cup U_k \rangle$$

\textbf{Определение 3.9.} Пусть $V$ -- линейное пространство, $U_1, ..., U_k \leq V$. Сумма этих подпространств называется \textit{прямой} (обозн.: $U = U_1 \oplus ... \oplus U_k$), если для любого вектора $u \in U$ существует единственный набор векторов $u_1 \in U_1, ..., u_k \in U_k$ такой, что $u = u_1 + ... + u_k$.

\textbf{Теорема 3.1.} Пусть $V$ -- линейное пространство, $U_1, ..., U_k \leq V$. Тогда сумма $U_1 + ... + U_k$ -- прямая $\Leftrightarrow$ для любого $i \in \{1, ..., k \}$ выполнено равенство:
$$U_i \cap (U_1 + ... + U_{i-1} + U_{i+1} + ... + U_k) = \{ \overline{0} \}$$

\textbf{Упражнение 3.3.} Доказать, что $n$-мерное линейное пространство является прямой суммой подпространства векторов, все координаты которых равны между собой, и подпространства векторов, сумма координат которых равна $0$.

\underline{Решение}: для краткости обозначим исходное пространство за $L_n$, первое подпространство за $U$, второе за $V$. Нужно проверить, что:
\begin{enumerate}[noitemsep]
    \item $U$ и $V$ -- это действительно подпространства.
    \item $U \cap V = \{ 0 \}$.
\end{enumerate}

\textbf{1.} Пусть $u \in U \Rightarrow u = (u_1, u_2, ..., u_n)^T = (u_1, u_1, ..., u_1)^T$ и $y \in U \Rightarrow$ $y = (y_1, y_1, ..., y_1)^T$. Очевидно, что сумма $u + y = (u_1 + y_1, ..., u_1 + y_1)^T$ лежит в $U$. Несложно проверить, что $\alpha u \in U \ \forall \alpha$. Значит $U$ -- действительно подпространство. Аналогично для $V$.

\textbf{2.} Рассмотрим $x \in U \cap V$. Так как $x \in U$, то $x = (x_1, ..., x_1)$. С другой стороны, $x \in V$ $\Rightarrow$ $\sum_{i=1}^n x_i = n x_1 = 0$ $\Rightarrow$ $x_1 = 0$ $\Rightarrow$ $x = (0, ..., 0)$. А значит $U \cap V = \{ 0 \}$.


\subsection*{Сопряжённое пространство}

\textbf{Определение 3.10.} Пусть $V$ -- линейное пространство. Множество линейных функционалов на $V$ называется \textit{пространством, сопряжённым к} $V$. Обозн.: $V^*$.

На $V^*$ определены операции сложения и умножения на скаляр:
\begin{itemize}[noitemsep]
    \item $\forall f, g \in V^* \ \forall v \in V \ (f+g)(v) = f(v) + g(v)$.
    \item $\forall \alpha \in \mathbb{R} \ \forall f \in V^* \ \forall v \in V \ (\alpha f)(v) = \alpha f(v)$.
\end{itemize}

\textbf{Утверждение 3.4.} $V^*$ также является линейным пространством.

Пусть $V$ -- линейное пространство, $e = (e_1, ..., e_n)$ -- базис в $V$. Тогда для каждого $i \in \{1, ..., k \}$ определим $f_i \in V^*$ следующим образом: $\forall v \in V: v = \sum_{i=1}^n \alpha_i e_i \rightarrow f_i(v) = \alpha_i$.

\textbf{Утверждение 3.5.} Пусть $V$ -- линейное пространство, $e = (e_1, ..., e_n)$ -- базис в $V$. Тогда $(f_1, ..., f_n)$ -- базис в $V^*$.

\textbf{Следствие.} Если $V$ -- линейное пространство, то dim $V^*$ = dim $V$. 

\textbf{Определение 3.11.} Пространством, \textit{дважды сопряжённым к} $V$ называется $V^{**} := (V^*)^*$. 

\textbf{Теорема 3.2.} Отображение $\varphi: V \rightarrow V^{**}$ такое, что $\varphi(v) = v^{**}$ для любого $v \in V$, является изоморфизмом линейных пространств $V$ и $V^{**}$.


\subsection*{Пространство линейных операторов}

\textbf{Определение 3.12.} Множество линейных отображений из $U$ в $V$ обозначается как $\mathcal{L} (U, V)$. Множество линейных преобразований пространства $V$ обозначается $\mathcal{L} (V)$.

\textbf{Упражнение 3.4.} Доказать, что  отображение $\psi: \mathcal{L} (U, V) \rightarrow M_{n \times k}(\mathbb{R})$, где $k = \dim U$, $n = \dim V$, биективно. 

По сути упражнение доказывает, что каждому линейному отображению из пространства $U$ размерности $k$ в пространство $V$ размерности $n$ соответствует своя матрица $A \in M_{n \times k}(\mathbb{R})$.

\end{spacing}
\end{document}