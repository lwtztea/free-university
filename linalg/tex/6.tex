\documentclass[a4paper, 12pt]{article}

\usepackage[T2A]{fontenc}
\usepackage[utf8]{inputenc}
\usepackage[english, russian]{babel}
\usepackage[left=19mm,right=19mm,top=2cm,bottom=2cm,bindingoffset=0cm]{geometry}
\usepackage{amsmath, amsfonts, amssymb, amsthm, mathtools}
\usepackage{enumitem}
\usepackage{setspace}
\usepackage{amsmath}
\usepackage{hyperref}

\begin{document}
\begin{spacing}{1.5}
\setlength{\parindent}{0ex}

\section*{Семинар 6}


\subsection*{6.1. Ранг матрицы}

\textbf{Определение 6.1.} Пусть $V$ -- линейное пространство, $X \subset V$. \textit{Рангом} системы $X$ называется наибольший размер линейно независимой подсистемы в $X$ (обозн. rk $X$).

\textbf{Утверждение 6.1.} rk $X = dim <X>$.

\textbf{Определение 6.2.} Пусть $A \in M_{n \times m}$.
\begin{itemize}[noitemsep]
    \item \textit{Строчным рангом} матрицы $A$ называется ранг $\text{rk}_r A$ системы её строк.
    \item \textit{Столбцовым рангом} матрицы $A$ называется ранг $\text{rk}_c A$ системы её столбцов.
\end{itemize}

\textbf{Теорема 6.1.} (о ранге матрицы). Для любой матрицы $A \in M_{n \times m}$ верно $\text{rk}_r A = \text{rk}_c A$.

Итак, можно говорить о \textit{ранге матрицы}, rk $A$, без уточнения, строчный он или столбцовый.

Вспомним одно из эквивалентных условий вырожденности матрицы -- строки или столбцы её линейно зависимы. Это значит, что для $A \in M_{n \times n}$ верно:

\textbf{Утверждение 6.2.} $A$ невырождена $\leftrightarrow$ rk $A = n$.

\textbf{Теорема 6.2.} Размерность пространства решений системы $Ax = 0$ равна $n-r$, где $n$ -- число неизвестных в системе, $r$ -- ранг основной матричной системы.

\textbf{Следствие.} В случае, когда $A \in M_{n \times n}$ и $A$ невырождена, решением $Ax = 0$ является только нулевой вектор (размерность пространства решений равна нулю).


\subsection*{6.2. Алгебраическая и геометрическая кратность собственного значения}

Вернёмся к характеристическому уравнению:
$$\chi_{\varphi}(\lambda) := \det (A - \lambda E) = 0$$
где $A \in M_{n \times n}$ -- матрица оператора $\varphi$. После разложения характеристического многочлена на множители получаем:
$$(\lambda - \lambda_1)^{k_1} \cdot ... \cdot (\lambda - \lambda_s)^{k_s} = 0$$
Тогда $\lambda = \lambda_i$ -- собственные значения, а $k_i$ -- их \textit{алгебраическая кратность}. Далее ищем собственные векторы из уравнения $Ax = \lambda_i x$.

Пусть мы нашли $r$ ЛНЗ собственных векторов, отвечающих собственному значению $\lambda_i$. Они порождают собственное подпространство $L_{\lambda_i} := < x_{\lambda_{i_1}}, ..., x_{\lambda_{i_r}} >$. Но как понять, что для $\lambda_i$ больше нет собственных векторов, линейно независимых с предыдущими? Очевидно (вообще говоря не очень и это нужно доказывать), что размерность $L_{\lambda_i}$ не может превосходить алгебраическую кратность $\lambda_i$, но они и не обязаны совпадать. На самом деле $\dim L_{\lambda_i} = n - \text{rk} (A - \lambda_i E)$, поскольку $L_{\lambda_i}$ есть не что иное, как пространство решений системы $(A - \lambda_i E)x = 0$. Число $\dim L_{\lambda_i}$ называется \textit{геометрической кратностью} $\lambda_i$.


\subsection*{6.3. Присоединённые векторы}

\textit{В качестве основы параграфа использовались \href{http://math.phys.msu.ru/archive/2017_2018/25/JF.pdf}{эти прекрасные конспекты}}.

Для начала рассмотрим линейный оператор с матрицей:
$$J_{k \times k} (\lambda_0) =
\begin{pmatrix}
\lambda_0 & 1 & 0 & \ldots & 0 & 0 \\
0 & \lambda_0 & 1 & \ldots & 0 & 0 \\
0 & 0 & \lambda_0 & \ldots & 0 & 0 \\
\vdots & \vdots & \vdots & \ddots & 0 & 0 \\
0 & 0 & 0 & \ldots & \lambda_0 & 1 \\
0 & 0 & 0 & \ldots & 0 & \lambda_0 \\
\end{pmatrix}
$$
Характеристический многочлен $\chi(\lambda) = (\lambda_0 - \lambda)^k$ имеет корень $\lambda_0$ кратности $k$. Теперь найдём размерность собственного подпространства, отвечающего нашему собственному значению. Для этого подставим $\lambda_0$ в $J - \lambda E$ и получим матрицу с единицами над главной диагональю. Нетрудно понять, что ранг такой матрицы равен $k - 1$ и, следовательно, размерность собственного подпространства равна $k - (k-1) = 1$. Это значит, что мы не сможем найти базис из $k$ ЛНЗ собственных векторов, чтобы привести $J_k (\lambda_0)$ к диагональному виду.

\textbf{Определение 6.3.} Матрица $J_k (\lambda_0)$ называется \textit{жордановой клеткой порядка} $k$, соответствующей собственному значению $\lambda_0$.

\textbf{Определение 6.4.} Ненулевой вектор $x$ называется \textit{присоединённым}, отвечающим собственному значению $\lambda$, если для некоторого $m \geq 1$ выполняется:
$$(A - \lambda E)^{m-1} x \neq 0, \ \ \ (A - \lambda E)^{m} x = 0.$$
Число $m$ называется \textit{высотой} присоединённого вектора $x$.

Далее я настоятельно прошу обратиться к конспектам по ссылке, потому что лучше, чем там, я вряд ли смогу сделать.

\end{spacing}
\end{document}