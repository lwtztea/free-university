\documentclass[a4paper, 12pt]{article}

\usepackage[T2A]{fontenc}
\usepackage[utf8]{inputenc}
\usepackage[english, russian]{babel}
\usepackage[left=19mm,right=19mm,top=2cm,bottom=2cm,bindingoffset=0cm]{geometry}
\usepackage{amsmath, amsfonts, amssymb, amsthm, mathtools}
\usepackage{enumitem}
\usepackage{setspace}
\usepackage{amsmath}

\begin{document}
\begin{spacing}{1.5}
\setlength{\parindent}{0ex}

\section*{Семинар 5}


\subsection*{5.1. Матрица линейного отображения}

Пусть $\textbf{e} = \{e_1, ... , e_n\}$ -- базис пространства $V$, а $\textbf{f} = \{ f_1, ... , f_m\}$ -- базис пространства $W$. Рассмотрим, куда перейдёт базисный вектор $e_j$ при его отображении $\varphi$ из $V$ в $W$:
$$\varphi e_j = \sum_{i=1}^m a_{ij} f_i$$
т.е. $\varphi e_j = (a_{1j}, ... , a_{mj})^T$ -- разложение вектора $\varphi e_j$ по базису $\textbf{f}$. Матрица, $j$-й столбец которой есть столбец координат $[\varphi e_j]_{\textbf{f}}$ называется \textit{матрицей отображения $\varphi$}, построенной в базисах $\textbf{e}$ и $\textbf{f}$. Получаем, что
$$[\varphi]_{\textbf{f,e}} = (a_{ij}) \in F^{m \times n}$$
-- линейное отображение однозначно восстанавливается по образам базисных векторов.

\textbf{Утверждение 5.1.} Для произвольного вектора $x \in V$ верно разложение:
$$[\varphi x]_{\textbf{f}} = [\varphi]_{\textbf{f,e}} [x]_{\textbf{e}},$$
где $[\varphi x]_{\textbf{f}}$ -- разложение $\varphi x$ по базису $\textbf{f}$, $[x]_{\textbf{e}}$ -- разложение $x$ по базису $\textbf{e}$, $[\varphi]_{\textbf{f,e}}$ -- матрица отображения $\varphi$, построенная в базисах $\textbf{e}$ и $\textbf{f}$.

\textbf{Доказательство.} Пусть $[x]_{\textbf{e}}= (\alpha_1, ..., \alpha_n)^T$. Далее для краткости записи опустим указание базисов. Перейдём к координатам $x$ и посмотрим на $\varphi x$:
$$\varphi x = \varphi \left ( \sum_{j=1}^n \alpha_j e_j \right ) = \sum_{j=1}^n \alpha_j \varphi e_j = \sum_{j=1}^n \alpha_j \sum_{i=1}^m a_{ij} f_i = \sum_{i=1}^m \left( \sum_{j=1}^n a_{ij} \alpha_j \right) f_i $$
Таким образом, мы действительно получили разложение вектора $\varphi x$ в базисе $\textbf{f}$, которое соответствует умножение матрицы $[\varphi]$ на координатный столбец $x$.


\subsection*{5.2. Детерминант матрицы}

На множестве квадратных матриц порядка $n$ задана числовая функция, если каждой матрице из этого множества сопоставлено некоторое число. Примерами могут служить:
\begin{itemize} [noitemsep]
    \item \textit{След} матрицы -- функция, сопоставляющая каждой квадратной матрице сумму ее диагональных элементов $a_{11} + a_{22} + ... + a_{nn}$.
    \item \textit{Евклидова норма} матрицы -- функция, сопоставляющая каждой вещественной матрице квадратный корень из суммы квадратов всех ее элементов.
\end{itemize}
Полезно ввести такую функцию, при помощи которой можно определить, является ли данная матрица вырожденной или нет.

\textbf{Определение 5.1.} Числовая функция $f$ на множестве квадратных матриц порядка $n$ называется \textit{детерминантом}, или \textit{определителем}, порядка $n$, а ее значение на матрице
$A$ -- \textit{детерминантом} $A$ (обозн. $\det A$), если она обладает следующими свойствами:
\begin{enumerate} [noitemsep]
    \item Какую бы строку матрицы A мы ни взяли, значение функции на матрице A является линейным однородным многочленом от элементов этой строки. Для i-й строки это значит, что $$f(A) = h_1 a_{i1} + h_2 a_{i2} + ... + h_n a_{in},$$ где $h_1, ..., h_n$ -- коэффициенты, не зависящие от элементов $i$-й строки $a_{i1}, ... , a_{in}$, но зависящие от остальных элементов матрицы.
    \item Значение функции на любой вырожденной матрице равно нулю.
    \item Значение функции на единичной матрице равно единице.
\end{enumerate}

\textbf{Утверждение 5.2.} Данная функция существует и единственна.

Основные свойства определителя:
\begin{itemize} [noitemsep]
    \item $\det A^T = \det A$
    \item $\det AB = \det A \det B$
    \item $A$ -- верхнетреугольная $\Rightarrow$ $\det A = a_{11} \cdot a_{22} \cdot ... \cdot a_{nn}$
\end{itemize}

\subsection*{5.3. Характеристический многочлен}

Зафиксируем базис и обозначим через $A$ матрицу линейного преобразования $\mathbf{A}$ в этом базисе. Тогда преобразование $\mathbf{A} - \lambda \mathbf{E}$ имеет матрицу $A - \lambda E$, и его ядро отлично от нуля тогда и только тогда, когда эта матрица вырождена. 

\textbf{Определение 5.2.} Для матрицы $A$ многочлен $\chi (\lambda) = \det (A - \lambda E)$ от $\lambda$ называют \textit{характеристическим многочленом} матрицы $A$, а уравнение $\chi (\lambda) = 0$ — \textit{характеристическим}.

\textbf{Утверждение 5.3.} Все корни характеристического многочлена матрицы являются её собственными значениями.

\textbf{Утверждение 5.4.} Характеристический многочлен не зависит от выбора базиса.

\textbf{Доказательство.} Пусть в пространстве $V$ имеется два базиса $\textbf{e}$ и $\textbf{f}$ и $T$ -- матрица перехода от $\textbf{e}$ к $\textbf{f}$. Матрицы оператора $\mathbf{A}$ в разных базисах связаны соотношением $A_{\textbf{f}} = T^{-1} A_{\textbf{e}} T$. Тогда:
\vspace{-0.4cm}
\begin{align*}
    \det (A_{\textbf{f}} - \lambda E) & = \det (T^{-1} A_{\textbf{e}} T - \lambda E) = \det (T^{-1} A_{\textbf{e}} T - \lambda T^{-1} E T) = \det \left [ T^{-1} (A_{\textbf{e}} - \lambda E) T \right ] = \\
    & = \det T^{-1} \det (A_{\textbf{e}} - \lambda E) \det T = \det (A_{\textbf{e}} - \lambda E) \det T^{-1} \det T = \det (A_{\textbf{e}} - \lambda E)
\end{align*}
Из этого утверждения следует, что мы можем назвать характеристический многочлен матрицы $A$ характеристическим многочленом линейного преобразования $\mathbf{A}$.


\subsection*{5.4. Мотивация дальнейших действий}

На предыдущем занятии мы рассмотрели инвариантные подпространства. Представим, что нам удалось найти базис $\textbf{e} = \{ e_1, ..., e_n \}$, в котором \textit{каждый} базисный вектор порождает инвариантное относительно некоторого преобразования $\varphi$ пространство. Как тогда будет выглядеть матрица этого преобразования в выбранном базисе? Давайте посмотрим, куда перейдёт базисный вектор $e_i$:
$$\varphi (e_i) = c_i \cdot e_i$$
поскольку пространство $<e_i>$ инвариантно относительно $\varphi$. Мы помним (я надеюсь на это), что в матрице линейного отображения $j$-ый столбец соответствует разложению вектора $\varphi e_j$ по базису $\textbf{f} = \{ f_1, ..., f_m \}$. В нашем случае это значит, что матрица $\varphi$ будет иметь диагональный вид и под действием этого преобразования каждый базисный вектор просто расстянется в $c_i$ раз.

Таким образом, если мы найдём этот базис и перейдём к нему, то сможем разложить любой вектор и легко понять, как будет выглядеть его образ под действием $\varphi$.

С другой стороны мы знаем, что собственные векторы порождают собственные подпространства, инвариантные относительно преобразования. Тогда будет здорово в качестве базиса выбрать систему ЛНЗ собственных векторов, а матрица преобразования будет диагональной, состоящей из собственных значений.

\end{spacing}
\end{document}
